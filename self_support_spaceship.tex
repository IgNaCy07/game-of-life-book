%%%%%%%%%%%%%%%%%%%%%%%%%%%%%%%%%%%%%%%%%%%%%%%%%%%%%%%%%%%%%%%%%%%%%%%%%
%%   CHAPTER: SELF-SUPPORTING SPACESHIPS
%%%%%%%%%%%%%%%%%%%%%%%%%%%%%%%%%%%%%%%%%%%%%%%%%%%%%%%%%%%%%%%%%%%%%%%%%

\renewcommand{\chapterfolder}{self_support_spaceships/}
\chapterimage{cover/self_support_spaceships}
\chapter{Self-Supporting Spaceships}\label{chp:self_support_spaceships}


\vspace*{-0.4in}
\epigraph{Life is too important a matter to be taken seriously.}{Oscar Wilde}
\vspace*{0.4in}


\noindent Universal computation, covered in the previous chapter, is the first of two major types of universality that often show up in discussions of Conway's Game of Life, and cellular automata in general. The second type of universality is universal construction. Whereas universal computation is the ability to ``compute anything that can be computed'', universal construction could be loosely summarized as the ability to ``construct anything that can be constructed''.

We will cover this topic in much more depth in Chapter~\ref{chp:universal_construction}, but this chapter provides a good preview of the general idea. In particular, we will construct patterns that can reach out into a region of empty space and construct something there.



%%%%%%%%%%%%%%%%%%%%%%%%%%%%%%%%
\section{Caterpillar}\label{sec:caterpillar}
%%%%%%%%%%%%%%%%%%%%%%%%%%%%%%%%

Stuff.


%%%%%%%%%%%%%%%%%%%%%%%%%%%%%%%%
\section*{Notes and Historical Remarks}
%%%%%%%%%%%%%%%%%%%%%%%%%%%%%%%%

Stuff.


%%%%%%%%%%%%%%%%%%%%%%%%%%%%%%%%%
\section*{Exercises \hfill \normalfont\textsf{\small solutions on \hyperlink{solutions_self_support_spaceships}{page \pageref{solutions_self_support_spaceships}}}}
\label{sec:solutions_self_support_spaceships}
\addcontentsline{toc}{section}{Exercises}
\vspace*{-0.4cm}\hrulefill\vspace*{-0.3cm}\footnotesize\begin{multicols}{2}\vspace*{-0.4cm}\raggedcolumns\interlinepenalty=10000
\setlength{\parskip}{0pt}
%%%%%%%%%%%%%%%%%%%%%%%%%%%%%%%%%


\begin{problem}\label{exer:self_support_spaceships_ex1}
	An exercise could be placed here.
\end{problem}


\mfilbreak


\begin{problem}\label{exer:self_support_spaceships_ex2}
	Another exercise could be placed here.
\end{problem}


%% EXERCISE END COMMANDS
\end{multicols}
\normalsize\vspace*{0.01cm}
%% DONE EXERCISE END COMMANDS