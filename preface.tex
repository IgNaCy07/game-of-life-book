\chapterimage{cover/preface.jpg} % Preface heading image
\renewcommand{\chapterfolder}{preface/}
\chapter{Wstęp}


\section*{Cel książki}\addcontentsline{toc}{section}{Cel książki}

W tej książce zapewniamy wprowadzenie do Gry w życie Conwaya, matematyki kryjącej się za nią, oraz metod użytych do stworzenia wielu z jej najciekawszych wzorów. Wiele wzorów-``cegiełek'' (zwłaszcza w pierwszych czterech rozdziałach książki) zostało znalezionych za pomocą brutalnej siły lub innymi sposobami przez komputer, i nie będziemy się rozwodzić na temat detali implementacyjnych owych wyszukiwań. Jednak od tamtego miejsca staramy się przeprowadzić czytelnika przez procesy myślowe i pomysły potrzebne do połączenia tych wzorów w bardziej interesujące i złożone konstrukcje.

Chociaż przeważnie książka podąża zgodnie z historią Gry w życie, to \emph{nie} jest to jednak jej główny cel. Jest to raczej efekt uboczny faktu, większość najnowszych odkryć bazuje na obiektach i technikach rozwiniętych wcześniej. Celem książki jest odarcie Gry w życie z jej tajemnic poprzez rozkładanie złożonych obiektów na osobne części możliwe do zrozumienia pojedynczo.

This book is up to date with regards to Life technology and results that were known as of January 15, 2022. However, new tools and techniques are discovered so frequently in the Game of Life that it will become somewhat out of date rather quickly.\footnote{There are at least three major discoveries mentioned in this book that were found between January 1--14, 2022.} A list of errata and notable discoveries since publication of the book can be found at \httpsurl{conwaylife.com/book}.

This book is up to date with regards to Life technology and results that were known as of January 15, 2022. However, new tools and techniques are discovered so frequently in the Game of Life that it will become somewhat out of date rather quickly.\footnote{There are at least three major discoveries mentioned in this book that were found between January 1--14, 2022.} A list of errata and notable discoveries since publication of the book can be found at \httpsurl{conwaylife.com/book}.


\section*{Grupa docelowa}\addcontentsline{toc}{section}{Grupa docelowa}

Mimo że ta książka nie ma żadnych formalnych wymagań matematycznych bądź informatycznych, jest ona skierowana do poziomu studenta pierwszego roku studiów licencjackich. Niektóre tematy z poziomu liceum jak logarytmy, funkcja ''podłoga'' $f(x) = \lfloor x \rfloor$ do zaokrąglania w dół, reprezentacja dwójkowa liczb naturalnych czy notacja sumowania jak $\Sigma_{k=1}^n k^2$ do dodawania liczb, są używane często i bez większych wyjaśnień. Najbardziej zaawansowaną matematyczną maszynerią jakiej będziemy używać będzie prawdopodobnie mnożenia matryc $2 \times 2$ w Sekcji~\ref{sec:pi_calc} (ale tylko górnych trójkątnych matryc $2 \times 2$, w dodatku z podanym wzorem na mnożenie matryc).

Jest również oczekiwane podstawowe rozumienie, jak działają dowody matematyczne i programowanie komputerów. To jest, oczekujemy od czytelnika pewnego poziomu dojrzałości matematycznej i informatycznej, lecz bez specjalistycznej wiedzy na żaden z tych tematów. W całej książce udowadniamy proste twierdzenia na temat Gry w Życie. Te dowody nie używają żadnych specjalistycznych technik dowodzenia ani nie wymagają od czytelnika żadnej specyficznej wiedzy matematycznej na poziomie uniwersytetu, tylko wymagają od czytelnika żeby był w stanie podążać za logicznym rozumowaniem. Podobnie, wprowadzimy język programowania do budowania programów komputerowych z obwodów z GwŻ w Rozdziale~\ref{chp:universal_computation}, żeby materiał był prostszy do opanowania jeżeli czytelnik zetknął się wcześniej z programowaniem.

Nieco bardziej zaawansowane elementy matematyki z jakich korzystamy są streszczone w Dodatku~\ref{app:math}, chociaż są one z reguły wprowadzone również w tekście głównym, zresztą jest potrzebne tylko ich powierzchowne zrozumienie. Używamy największych wspólnych dzielników, najmniejszych wspólnych wielokrotności i tożsamości B\'ezouta przy omawianiu okresów oscylatorów w Rozdziale~\ref{chp:oscillators}, zatem wprowadzamy te narzędzia w Dodatku~\ref{sec:gcd}. Nieskończone ciągi pojawiają się na krótko pod koniec Rozdziału~\ref{chp:periodic_circuitry} i w Sekcji~\ref{sec:pi_calc}, ale nie oczekujemy od czytelników jakiejś wiedzy na ten temat ani znajomości problemów ze zbieżnością. Wreszcie, użyjemy notacji dużego $\Theta$ do omówienia wielkości i tempa wzrostu wzorów w Sekcjach~\ref{sec:large_glider_guns}, \ref{sec:irreg_guns} i~\ref{sec:Osqrtlogt}, zatem ten koncept jest wyjaśniony w Dodatku~\ref{sec:bigO}.


\section*{Jak korzystać z książki}\addcontentsline{toc}{section}{Jak korzystać z książki}

Gra w życie Conwaya jest niezmiernie wizualna, wobec czego ta książka bardzo liberalnie korzysta ze sporej ilości ilustracji, zwłaszcza gdy są wprowadzane nowe wzory lub techniki ich budowania. Gra w życie jest jednak obserwowana w ruchu, przez co nieruchome obrazy nie są idealnym narzędziem do nauki. Aby lepiej przedstawić wzory w ruchu, stosujemy pięć metod:\medskip

\begin{itemize}
	\item Ilustracje ukazują żywe komórki na czarno, a nieżywe na biało, ale używamy też gradientu z niebieskiego do pomarańczowego, aby oznaczyć komórki będące żywe w przeszłych generacjach. Jasnoniebieskie komórki były żywe przed chwilą, zaś pomarańczowe w nieco bardziej odległej przeszłości (około $75$~generacji lub więcej---zobacz Ryc.~\ref{fig:preface_lwss}).\bigskip

	\noindent\begin{minipage}{\linewidth}
		\centering\patternimglink{0.125}{lwss}
		\captionof{figure}{An object moving to the right with a gradient behind it indicating how long ago it was in each location. White cells were never alive, black cells are currently alive, blue cells were alive recently, and orange cells were alive long ago.}\label{fig:preface_lwss}\bigskip
	\end{minipage}
	
	\item If we really wish to emphasize what a pattern looks like in different generations, all generations of interest will be displayed, along with arrows that specify how many generations have passed. For example, if we want to clarify exactly what the pattern in Figure~\ref{fig:preface_lwss} does as it moves from left to right, we might display it as in Figure~\ref{fig:preface_lwss_2}.\bigskip
	
	\noindent\begin{minipage}{\linewidth}
		\centering\patternlink{lwss}{\vcenteredhbox{\patternimg{0.125}{lwss_1}} \vcenteredhbox{\genarrow{1}} \vcenteredhbox{\patternimg{0.125}{lwss_2}} \vcenteredhbox{\genarrow{1}} \vcenteredhbox{\patternimg{0.125}{lwss_3}} \vcenteredhbox{\genarrow{1}} \vcenteredhbox{\patternimg{0.125}{lwss_4}} \vcenteredhbox{\genarrow{1}} \vcenteredhbox{\patternimg{0.125}{lwss_5}}}
		\captionof{figure}{The same object as in Figure~\ref{fig:preface_lwss}, but displayed in a more explicit fashion. This image shows how the pattern changes every time 1 generation elapses.}\label{fig:preface_lwss_2}\bigskip
	\end{minipage}

	\item We use colors to highlight various pieces of patterns, and we maintain a consistent coloring scheme throughout the book. Light pastel colors like aqua, magenta, light green, yellow, and light orange are used to highlight \emph{around} objects---cells in these colors are dead, and they indicate a region in the Life plane consisting of cells that all serve some common purpose or logically make up one ``object'' (see Figure~\ref{fig:preface_p15_gun}). On the other hand, darker colors like dark green, dark orange, and dark red are used to highlight certain live cells. Dark green is typically used to highlight the input to some reaction, while dark orange is typically used for the output of the reaction (see Figure~\ref{fig:preface_fx77_p5_eat}).\bigskip
	
	\noindent\begin{minipage}{\linewidth}
		\centering
		\begin{minipage}[b]{0.56\textwidth}
			\centering
			\patternimglink{0.11755364806}{p15_gun}
			\captionof{figure}{A glider gun made up of several different component reactions that are highlighted in different colors. In particular, gliders are created by a gun highlighted in \bgbox{magentaback}{magenta}, lightweight spaceships are created by guns highlighted in \bgbox{aquaback}{aqua} and \bgbox{yellowback2}{yellow}, and those lightweight spaceships are merged into the original glider stream by oscillators highlighted in \bgbox{orangeback2}{light orange}.}\label{fig:preface_p15_gun}
		\end{minipage}\hfill
		\begin{minipage}[b]{0.4\textwidth}
			\centering
			\patternimglink{0.11}{fx77_p5_eat}
			\captionof{figure}{A \bgbox{greenback}{dark green} Herschel comes in from the left and creates the \bgbox{orangeback}{dark orange} Herschel on the right. Cells highlighted in blue are constantly changing due to being part of an oscillator, whereas cells that are light orange are usually dead, except when the Herschel passes through.}\label{fig:preface_fx77_p5_eat}
		\end{minipage}\bigskip
	\end{minipage}
	
	\item In the electronic version of this book, almost every figure is actually a clickable link that will open a text file containing RLE or Macrocell code for the displayed \ifdefined\FORPRINTING pattern.\else pattern (go ahead, click on any of the figures above, or even one of the five images on the cover page).\fi\ This code can be copied and pasted into Life simulation software like Golly (\httpurl{golly.sourceforge.net}) so that it can be explored and manipulated.\footnote{For an explanation of how RLE or Macrocell code works, see \httpsurl{conwaylife.com/wiki/Run_Length_Encoded} or \httpsurl{conwaylife.com/wiki/Macrocell}.} Note that clicking on figures may not work in certain PDF viewers (such as the viewers built into web browsers), so we recommend using Adobe Acrobat Reader (\httpurl{get.adobe.com/reader}) to read this book digitally.\smallskip
	
	\item RLE, LifeHistory, or Macrocell codes for all of the patterns displayed in the book are also available at the book's website (\httpsurl{conwaylife.com/book}). Furthermore, the patterns can be viewed and manipulated right on that website as well via any modern web browser, without downloading any additional software.\medskip
\end{itemize}


\subsection*{Exercises}

We strongly encourage the reader to work through this book's many exercises that can be found at the end of each chapter. For this reason, it is extremely important to either download Life software or use the book's website to view and edit patterns while making your way through the book---Life is meant to be played, not just watched, and many of the exercises simply cannot be solved without the assistance of Life simulation software.

Roughly half of the exercises are marked with an asterisk ($\ast$), which means that they have a solution provided in Appendix~\ref{chp:solutions}. Exercises also begin with a numeric difficulty estimate on a scale of 1 to 5, displayed within square brackets like \probdiff{3} or \probdiff{5}. Difficulty-1 exercises follow fairly immediately from the relevant definitions and results in the text, up to difficulty-4 exercises that involve either an extensive and delicate calculation, or a complicated multi-step construction. Difficulty-5 exercises are more like projects that could take a day or so of work to complete.


\section*{Acknowledgments}\addcontentsline{toc}{section}{Acknowledgments}

The authors are indebted to dozens of people who opened the world of Conway's Game of Life to them. Rather than acknowledging the people who discovered the reactions and patterns that we discuss here, they are credited in footnotes and figures throughout the book.

We extend thanks to Adam P.~Goucher for numerous contributions to the writing of this book, especially in Chapter~\ref{chp:0e0p}. Thanks to Nicolay Beluchenko, Oscar Cunningham, Steven Eker, Bill Gosper, Hartmut Holzwart, Tanner Jacobi, Matthias Merzenich, Daniel Mouscher, Mark Niemiec, David Raucci, Chris Rowett, Ville Salo, Rich Schroeppel, Michael Simkin, Connor Steppie, Satoshi Tanaka, Justin Tang, and Kalan Warusa, as well as ``AlephAlpha'', ``Chris857'', ``dani'', ``Ian07'', and many other ConwayLife.com forum users, for helpful conversations and corrections to early versions of this book.

Thanks to Andrew Trevorrow and Tomas Rokicki for creating the open-source cross-platform cellular automaton editor and simulator Golly (\httpurl{golly.sourceforge.net}), without which many of the patterns discussed in this book would not have been discovered. Thanks to Chris Rowett for creating LifeViewer (\httpsurl{lazyslug.com/lifeviewer}), which is used on this book's website (\httpsurl{conwaylife.com/book}) to display patterns and make them interactive. Thanks to Velimir Gayevskiy and Mathias Legrand for the \emph{Legrand Orange Book} LaTeX template from \httpurl{LaTeXTemplates.com}.

Finally, the authors would like to thank their wives Kathryn and Melanie for tolerating them during their years of mental absence glued to this book, and their parents for encouraging them to care about both learning and teaching. Many thanks also to Mount Allison University for giving the first author the academic freedom to pursue a project like this one.
% Note about license of Legrand template: https://latex.org/forum/viewtopic.php?f=59&t=30823&p=104225#p104225
