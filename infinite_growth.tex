%%%%%%%%%%%%%%%%%%%%%%%%%%%%%%%%%%%%%%%%%%%%%%%%%%%%%%%%%%%%%%%%%%%%%%%%%
%%   CHAPTER: INFINITE GROWTH
%%%%%%%%%%%%%%%%%%%%%%%%%%%%%%%%%%%%%%%%%%%%%%%%%%%%%%%%%%%%%%%%%%%%%%%%%

\renewcommand{\chapterfolder}{infinite_growth/}
\chapterimage{infinite_growth/cover}
\chapter{Infinite Growth}\label{chp:infinite_growth}\index{infinite growth}


\vspace*{-0.4in}
\epigraph{In the vast infinity of life, all is perfect, whole, and complete.}{Louise L. Hay}
\vspace*{0.4in}

We have several patterns that grow arbitrarily large over the past few chapters: the Gosper glider gun from Section~\ref{sec:queen_bee} and the new gun from Section~\ref{sec:twin_bees} both grow at a linear rate (i.e., their population in generation $n$ is proportional to $n$) by shooting off never-ending streams of gliders, and the breeder from Section~\ref{sec:gosper_breeder} grows quadratically (i.e., its population in generation $n$ is proportional to $n^2$) by creating Gosper glider guns.

In this chapter, we will investigate other methods for creating infinitely-growing patterns with unusual properties. For example, what is the smallest pattern that leads to infinite growth? What is the smallest pattern that leads to quadratic growth? Can we find patterns that grow even slower than linearly, or pattern whose growth rate is between linear and quadratic?

% Switch engines and how to synthesize them. Gives new easy breeders
% Different types of breeders in general. Mention Herschel track breeder in general
% Mention spacefillers, and note that they are the fastest way to fill the plane (density 1/2 we already proved optimal, and c/2 in each direction we alreayd proved optimal).
% Growth rates between linear and quadratic.


\section{Linear Growth}\label{sec:linear_growth}

We say that a pattern exhibits \emph{linear growth} if there exist positive constants $m$ and $M$ such that its population in generation $n$ is always between $mn$ and $Mn$. We have already seen several different types of patterns that grow linearly: glider guns, puffers, and rakes are all examples of patterns with this type of growth. And in fact, they all demonstrate fundamentally different types of linear growth, which we categorize as follows:

\begin{itemize}
	\item \textbf{SM}: A \textbf{s}tationary object creates infinitely many \textbf{m}oving objects. Glider guns are by far the most common and well-known objects in this category.
	
	\item \textbf{MS}: A \textbf{m}oving object creates infinitely many \textbf{s}tationary objects. Puffers like the block-laying switch engine fall into this category, and wickstretchers [?? INTRODUCE WHEN] also fit more cleanly here than elsewhere.
	
	\item \textbf{MM}: A \textbf{m}oving object creates infinitely many \textbf{m}oving objects. In other words, this category consists of rakes.
	
	\item \textbf{SS}: A \textbf{s}tationary object creates infinitely many \textbf{s}tationary objects. The existence of such a pattern is perhaps a bit difficult to visualize.
\end{itemize}

% SS does exist: http://conwaylife.com/forums/viewtopic.php?f=2&t=496#p3278
% sqrtgun: http://radicaleye.com/DRH/sqrtgun10.1.html


\subsection{Wickstretchers}\label{sec:wickstretchers}

Stuff.


\section{Sawtooths}\label{sec:sawtooths}\index{sawtooth}

Most patterns that we have seen so far in this book have a more-or-less monotonic population---they (for the most part) increase

% different methods such as tractor beams here. have a catalog of tractor beams, exercises to make sawtooths with them. Show in the main body the tractor beam used in slow growth  http://conwaylife.com/forums/viewtopic.php?f=2&t=1649#p20951


\section{Breeders and Quadratic Growth}\label{sec:breeders}\index{breeder}

A \emph{breeder}\index{breeder} is a pattern whose population grows quadratically. We constructed our first breeder in Section~\ref{sec:gosper_breeder}, and it worked by using \textbf{m}oving objects (space rakes) to create \textbf{s}tationary objects (Gosper glider guns) that create \textbf{m}oving objects (gliders). Thus we say that it is an MSM breeder. However, there are many different ways that a breeder could work, and we categorize them as follows:
\begin{itemize}
	\item SMM breeder: A \textbf{s}tationary object produces \textbf{m}oving objects that produces \textbf{m}oving objects. In other words, this is a stationary object that creates rakes, and is thus sometimes called a \emph{rake gun}\index{rake gun}.
	\item MSM breeder: A \textbf{m}oving object creates \textbf{s}tationary objects that create \textbf{m}oving objects. The breeder that we already constructed is of this type.
	\item MMS breeder: A \textbf{m}oving object produces \textbf{m}oving objects that create \textbf{s}tationary objects. That is, breeders in this category work by using a collection of puffers and rakes to create additional puffers as they move.
	\item MMM breeder: A \textbf{m}oving object creates \textbf{m}oving objects that create more \textbf{m}oving objects. In other words, this is a collection of puffers and rakes that create additional rakes as they move.
\end{itemize}

We will now construct breeders of the three types (SMM, MMS, and MMM) that we have not yet seen.


\subsection{An MMS Breeder}\label{sec:mms_breeder}

We start with an MMS breeder, since its construction will actually be fairly straightforward thanks to the fact that some MS objects (i.e., puffers) that are easy to construct with gliders: the block-laying and glider-producing switch engines.\index{switch engine!block-laying}\index{switch engine!glider-producing} In particular, since we know a 4-glider synthesis of the switch engine (see Table~\ref{tab:other_glider_synth}) and a 2-glider synthesis of a block (see Table~\ref{tab:2_glider_synth}), and we saw in Figure~\ref{fig:switch_engine_block} that a block together with a switch engine can create a block-laying switch engine, so we have all of the ingredients we need to use $6$ gliders to synthesize a puffer, as in Figure~\ref{fig:blse_synth}.
\begin{figure}[!ht]
	\centering\includegraphics[scale=0.1]{infinite_growth/blse_synth.png}
	\caption{A six-glider synthesis of a block-laying switch engine. The two gliders on the right (outlined in \bgbox{yellowback2}{yellow}) create a block, while the four gliders on the left (outlined in \bgbox{aquaback}{aqua}) create a switch engine. When the switch engine hits the block, it creates a block-laying switch engine, as in Figure~\ref{fig:switch_engine_block}.}\label{fig:blse_synth}
\end{figure}

With this synthesis of the block-laying switch engine in hand, all we have to do is use rakes to shoot gliders in the proper locations. Since the block-laying switch engine is a fairly wide object, the period~16, 20, and 60 rakes that we constructed do not space their glider waves apart far enough to work. However, the period~80 rakes from Figure~\ref{fig:coe_space_rake_stabilized} space their glider waves 40~cells apart, which is wide enough for the block-laying switch engines to survive, and we can simply piece everything together as in Figure~\ref{fig:blse_breeder}.
\begin{figure}[!ht]
	\centering\includegraphics[width=0.95\textwidth]{infinite_growth/blse_breeder.png}
	\caption{A breeder that uses two backward period~80 rakes (outlined in \bgbox{yellowback2}{yellow}) to synthesize blocks and four more period~80 rakes (outlined in \bgbox{aquaback}{aqua}) to synthesize switch engines. When the switch engines hit the blocks, they become block-laying switch engines, and hence this pattern grows quadratically in such a way that it fills a triangular region of the plane with blocks.}\label{fig:blse_breeder}
\end{figure}

This breeder moves to the right and constructs an ever-expanding line of block-laying switch engines that move to the top-right, each leaving a trail of blocks behind them. Thus this breeder creates a triangular region filled with blocks (see Figure~\ref{fig:blse_breeder_1600}), just like our previous breeders created triangular regions filled with gliders. The main difference with this breeder is that the objects that make up the triangle are themselves stationary, but the mechanism that expands the triangle moves.
\begin{figure}[!ht]
	\centering\includegraphics[width=0.95\textwidth]{infinite_growth/blse_breeder_1600.png}
	\caption{After running the breeder in Figure~\ref{fig:blse_breeder} for 1,600~generations, we see a triangular region of blocks form on the left, with a line of block-laying switch engines stretching that triangle to the top-right. As the breeder moves farther to the right, the triangle continues to expand.}\label{fig:blse_breeder_1600}
\end{figure}


\subsection{An SMM Breeder}\label{sec:smm_breeder}

As our example of an SMM breeder, we will construct a gun that fires rakes. The space rake and the rakes based on the Coe ship both require roughly the same effort to synthesize, so we arbitrarily choose to construct a space rake gun. Since space rakes cannot be separated by fewer than $86$ generations without crashing into each other, we will use the period~$92$ glider gun that we constructed in Figure~\ref{fig:p92_gun} as our primary building block in this breeder.\footnote{We could use period~$86$ glider guns instead, but the smallest known period~$86$ gun is quite large, so we take the trade-off of having a slightly higher period but a smaller gun. We will also see how to construct a small period~$90$ glider gun in Section~\ref{sec:p30}, which would be another good option for this breeder.} To start with, the space rake makes use of several lightweight spaceships, so we will need to construct a period~$92$ lightweight spaceship gun. This is straightforward to do by making use of a $3$-glider synthesis of the LWSS, as demonstrated in Figure~\ref{fig:p92_lwss_gun} (since we have constructed LWSS guns from glider guns a few times already, this type of construction is becoming standard for us, so this will be the last time that we explicitly point out how to do it).
\begin{figure}[!ht]
	\centering\includegraphics[scale=0.1]{infinite_growth/p92_lwss_gun.png}
	\caption{A period~$92$ lightweight spaceship gun, constructed by aiming three period~$92$ glider guns (outlined in \bgbox{aquaback}{aqua}, \bgbox{yellowback2}{yellow}, and \bgbox{magentaback}{magenta}) so as to synthesize a stream of lightweight spaceships (outlined in \bgbox{greenback}{green}).}\label{fig:p92_lwss_gun}
\end{figure}

We can use this lightweight spaceship gun to complete the first two steps in the sequential synthesis of the ecologist provided in Figure~\ref{fig:ecologist_synth}. Next, we have to use two gliders to create a block and four gliders to turn that block into a LWSS and B-heptomino. This is the most difficult part of the synthesis, since we need to merge some glider streams to get the six gliders (three travelling in each of two different directions) close enough together to work. In particular, we will use the technique that we introduced in Figure~\ref{fig:lwss_reflect_glider} for using an LWSS to insert a glider close to another glider. Half of the mechanism that synthesizes the LWSS and B-heptomino is displayed in Figure~\ref{fig:p92_lwss_inserter_for_breeder}: one period~$92$ glider stream is initially generated, and two new period~$92$ glider streams are then interspersed with it. The other half of the six-glider collision (i.e., the stream containing the other three gliders) is created in a very similar way.
\begin{figure}[!ht]
	\centering\includegraphics[width=\textwidth]{infinite_growth/p92_lwss_inserter_for_breeder.png}
	\caption{A mechanism that produces three gliders every $92$ generations, packed closely together so as to form half of the synthesis of the LWSS and B-heptomino combination that is needed to complete the synthesis of the ecologist (a gun that produces the other half of the LWSS and B-heptomino synthesis is extremely similar and thus is not shown here). The \bgbox{magentaback}{magenta} glider stream synthesizes a block, the \bgbox{aquaback}{aqua} glider stream turns that block into an LWSS, and the \bgbox{yellowback2}{yellow} glider stream synthesizes the B-heptomino. The precise orientation and spacing of three gliders in nearby streams is attained by using the reaction of Figure~\ref{fig:lwss_reflect_glider} to turn a LWSS and a glider into a perpendicular glider, and the lightweight spaceships used in that reaction are created by the LWSS gun from Figure~\ref{fig:p92_lwss_gun}.}\label{fig:p92_lwss_inserter_for_breeder}
\end{figure}

At this point, we have created a period~$92$ gun that shoots ecologists. To turn it into a space rake gun, we need to perform the final step of its sequential synthesis, which is to synthesize a block and then turn that block into another lightweight spaceship. This requires $4$ gliders ($2$ coming from each of $2$ different directions), so we again use the glider-inserting reaction of Figure~\ref{fig:lwss_reflect_glider}. Since the mechanism that creates these close gliders uses the exact same tricks as in Figure~\ref{fig:p92_lwss_inserter_for_breeder}, we do not display it explicitly on its own. Instead, we jump directly to the finished product, which is the rather large period~$92$ space rake gun (and thus an SMM~breeder) shown in Figure~\ref{fig:space_rake_gun}.\footnote{Dieter Leithner used the same synthesis and many of the same ideas in the construction of this breeder to construct a more compact period~$90$ space rake gun in November 1994. The precise technology that he used was based on manipulating period~$30$ glider streams, which we will investigate in Section~\ref{sec:p30}.}
\begin{figure}[!ht]
	\centering\includegraphics[width=\textwidth]{infinite_growth/space_rake_gun.png}
	\caption{A period~$92$ gun that creates space rakes.}\label{fig:space_rake_gun}
\end{figure}


\subsection{An MMM Breeder}\label{sec:mmm_breeder}

Stuff.



% SSM breeder: slide-breeder
% SMS breeder: http://conwaylife.com/forums/viewtopic.php?f=2&t=524#p3526 (by Paul Tooke)
% MSS breeder: http://conwaylife.com/forums/viewtopic.php?f=2&t=496&start=25#p3380 (also Paul Tooke)
% SSS breeder here: http://conwaylife.com/forums/viewtopic.php?f=2&t=524 by Paul Tooke, but grows sub-quadratically


\section{Spacefillers}\label{sec:spacefillers}\index{spacefiller}

Stuff.


\section{Unusual Growth Rates}\label{sec:growth_rates}

Stuff.
% sqrtsqrtgun: http://conwaylife.com/forums/viewtopic.php?f=2&t=1649#p20958
% O(t^1.5) etc
% Backref to life computes pi for irrational growth rate


\section{Small Infinite Growth}\label{sec:small_infinite_growth}

Stuff.


\section{Unknown (In)Finite Growth}\label{sec:unknown_growth}

There are also several methods known for constructing patterns which we don't even know if they continue to grow forever or not.

[FERMAT PRIME CALCULATOR, ETC]


\subsection{Line Puffers}\label{sec:line_puffers}

The previous examples perhaps seem somewhat unsatisfying, since their unknown eventual fates all come down to the fact that they encode difficult mathematical problems. Somehow these patterns do not seem very ``natural''; the difficulty in determining their eventual fate feels rather cooked up.

However, it should not be difficult to believe that there are rather simple patterns with difficult-to-determine long-term behaviour as well. After all, the 19-cell pattern from way back in Figure~\ref{fig:exercise_ark} evolved in a somewhat unpredictable manner for over 6.5 million generations before stabilizing.

One type of object that is particularly good at being difficult to analyze is a \emph{line puffer}:\index{line puffer} a puffer whose back end (in at least one of its phases) is a straight line of alive cells perpendicular to the direction of travel of the puffer.

[GIVE SIMPLE EXAMPLE]


%%%%%%%%%%%%%%%%%%%%%%%%%%%%%%%%%
\section*{Exercises \hfill \normalfont\textsf{\small solutions on \hyperlink{solutions_infinite_growth}{page \pageref{solutions_infinite_growth}}}}
%%%%%%%%%%%%%%%%%%%%%%%%%%%%%%%%%

\begin{problem}\label{exer:gpsw_breeder_3_glider}
	Use 3 rakes and the 3-glider synthesis of the glider-producing switch engine from Table~\ref{tab:3_glider_synth} to create a breeder. [Hint: None of the period~16, 20, 60, or 80 rakes that we have constructed will work, since their glider waves are too close together. Use a higher-period rake, like the one from Exercise~\ref{exer:p240_rake}.]
\end{problem}
% x = 272, y = 203, rule = B3/S23
% 261b2o5b4o$259b2ob2o3bo3bo$259b4o8bo$260b2o5bo2bo2$258bo$257b2o8b2o$
% 256bo9bo2bo$257b5o4bo2bo$258b4o3b2ob2o$261bo4b2o2$244bobo$245b3o$245b
% 3o20b4o$267bo3bo$250b4o17bo$232bo16bo3bo13bo2bo$230b2o21bo$231b2o16bo
% 2bo$241bo2bo$245bo$228b2o11bo3bo$230bo11b4o$225bo7b2o2b3o$224bo8b2o2b
% 3o$225bo7b2o2b3o$230bo11b4o$228b2o11bo3bo$245bo$241bo2bo2$187bo43b2o$
% 185b2o42b2ob2o$161b2o23b2o41b4o$159b2ob2o66b2o$159b4o$160b2o4$163b2o
% 31b2o$163bobo26b4ob2o$163bo28b6o$193b4o5b4o$175b2o24bo3bo$175bobo27bo$
% 175bo14bo13bo$162b2o24bo2bo9b2o$160b2ob2o22bo$160b4o23bo9bob2o3b2o$
% 161b2o24bobo7b3ob3ob2o$199b7o$201b4o3$191b6o$190bo5bo$196bo$190bo4bo$
% 192b2o2$97bo$95b2o$96b2o6$177b2o$168b4o4b4o$167bo3bo4b2ob2o$171bo6b2o$
% 167bo2bo2$165b2o$165b2o8b2o$164bo3bo5bo2bo$165bo3bo4bo2b2o$165bo4bo2bo
% 2b2o$167bobo3b4o2$154b2o$152bo24b2o$153bobo20b4o$154bo4b2o15b2ob2o$
% 158b4o16b2o$139bo18b2ob2o$138bo21b2o$138b3o2$152b2o$137bo12b2ob2o$137b
% o8bo3b4o$141b2o2bobo3b2o$132b2o6bo2b2o3bo$141b2o2bobo3b2o$137bo8bo3b4o
% $137bo12b2ob2o$152b2o3$94bo43b4o$93bo43bo3bo$68b4o21b3o45bo$67bo3bo65b
% o2bo$71bo$67bo2bo4$71b2o28b6o$70b2o28bo5bo$72bo33bo4b2o$100bo4bo4b4o$
% 83b2o17b2o6b2ob2o$82b2o28b2o$84bo$69b4o$68bo3bo22b2o11b2o$72bo21b2o9bo
% b8o$68bo2bo24bo8bo8bo$107bo6bo$113bo$110b2o$100b4o$99b6o$99b4ob2o$103b
% 2o4$4bo$3bo$3b3o$bo$2o$obo5$97b6o$96bo5bo$102bo$96bo4bo$98b2o2$110b2o$
% 92bo11bob4ob2o$68b2o21bo11b2o2b5o$66b2ob2o20b3o10bo3b3o$66b4o$67b2o11b
% o$79bo28b4o$79b3o25bo3bo$111bo$68bo33b2o3bo2bo$67bo30b4ob2o$67b3o28b6o
% $99b4o4$67b2o$65b2ob2o21bo$65b4o21b2o45b2o$66b2o22bobo42b2ob2o$135b4o$
% 136b2o2$148b4o$147bo3bo$143bo7bo$138b7o2bo2bo$137bo2b3ob2o$138b7o2bo2b
% o$143bo7bo$147bo3bo$148b4o$136bo$135b2o$135bobo18b4o$155bo3bo$159bo14b
% 4o$151bo3bo2bo14bo3bo$150b2o25bo$150bobo20bo2bo2$171b2o$170b5o$163bob
% 3o2bo4bo$161b2o3bo3b3o2bo$161bobo7bo2b2o$161bob2o7b2o$162b2o$163bo2$
% 167b2o5b4o$165b2ob2o3bo3bo$165b4o8bo$166b2o5bo2bo!

\begin{problem}\label{exer:blse_breeder_triangle}
	We saw in Figure~\ref{fig:blse_breeder_1600} that our block-laying switch engine breeder creates a triangular region filled with blocks. What are the angles between the sides of this triangle? [Hint: One angle is easy to figure out. Use the law of cosines and the law of sines to determine the rest.]
	% SOLUTION:
	% bottom-left: pi/4
	% top-center: arcsin(3/sqrt(13))
	% bottom-right: arcsin(1/sqrt(26))
	% side lengths are (up to scaling) sqrt(26)/6, 1, and 1/(3*sqrt(2))
\end{problem}


% Exercise: Use "max" components to create a greyship